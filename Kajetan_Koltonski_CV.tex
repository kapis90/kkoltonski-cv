%% start of file `template.tex'.
%% Copyright 2006-2015 Xavier Danaux (xdanaux@gmail.com), 2020-2022 moderncv maintainers (github.com/moderncv).
%
% This work may be distributed and/or modified under the
% conditions of the LaTeX Project Public License version 1.3c,
% available at http://www.latex-project.org/lppl/.

\documentclass[10pt,letterpaper,sans]{moderncv}        % possible options include font size ('10pt', '11pt' and '12pt'), paper size ('a4paper', 'letterpaper', 'a5paper', 'legalpaper', 'executivepaper' and 'landscape') and font family ('sans' and 'roman')

% moderncv themes
\moderncvstyle{classic}                            % style options are 'casual' (default), 'classic', 'banking', 'oldstyle' and 'fancy'
\moderncvcolor{blue}                               % color options 'black', 'blue' (default), 'burgundy', 'green', 'grey', 'orange', 'purple' and 'red'
%\renewcommand{\familydefault}{\sfdefault}         % to set the default font; use '\sfdefault' for the default sans serif font, '\rmdefault' for the default roman one, or any tex font name
\photo[3cm][1pt]{../profile.jpg}
\usepackage{amsmath} % for math

% adjust the page margins
\usepackage[scale=0.75]{geometry}
\setlength{\hintscolumnwidth}{3.8 cm}                % if you want to change the width of the column with the dates
%\setlength{\makecvheadnamewidth}{10cm}            % for the 'classic' style, if you want to force the width allocated to your name and avoid line breaks. be careful though, the length is normally calculated to avoid any overlap with your personal info; use this at your own typographical risks...

% font loading
% for luatex and xetex, do not use inputenc and fontenc
% see https://tex.stackexchange.com/a/496643
\ifxetexorluatex
  \usepackage{fontspec}
  \usepackage{unicode-math}
  \defaultfontfeatures{Ligatures=TeX}
  \setmainfont{Latin Modern Roman}
  \setsansfont{Latin Modern Sans}
  \setmonofont{Latin Modern Mono}
  \setmathfont{Latin Modern Math} 
\else
  \usepackage[T1]{fontenc}
  \usepackage{lmodern}
\fi

% document language
\usepackage[english]{babel}  % FIXME: using spanish breaks moderncv

% personal data
\name{Kajetan Kołtoński}{}
% \familyname{}
\address{Przeciszòw}{}
\phone[mobile]{+48 537 648 594}
\email{k.koltonski@gmail.com}

\social[linkedin]{yourusername}
\social[github]{yourusername}
% Social icons
% \social[linkedin]{john.doe}                        % optional, remove / comment the line if not wanted
% \social[xing]{john\_doe}                           % optional, remove / comment the line if not wanted
% \social[twitter]{ji\_doe}                          % optional, remove / comment the line if not wanted
% \social[github]{jdoe}                              % optional, remove / comment the line if not wanted
% \social[gitlab]{jdoe}                              % optional, remove / comment the line if not wanted
% \social[stackoverflow]{0000000/johndoe}            % optional, remove / comment the line if not wanted
% \social[bitbucket]{jdoe}                           % optional, remove / comment the line if not wanted
% \social[skype]{jdoe}                               % optional, remove / comment the line if not wanted
% \social[orcid]{0000-0000-000-000}                  % optional, remove / comment the line if not wanted
% \social[researchgate]{jdoe}                        % optional, remove / comment the line if not wanted
% \social[researcherid]{jdoe}                        % optional, remove / comment the line if not wanted
% \social[telegram]{jdoe}                            % optional, remove / comment the line if not wanted
% \social[whatsapp]{12345678901}                     % optional, remove / comment the line if not wanted
% \social[signal]{12345678901}                       % optional, remove / comment the line if not wanted
% \social[matrix]{@johndoe:matrix.org}               % optional, remove / comment the line if not wanted
% \social[googlescholar]{googlescholarid}            % optional, remove / comment the line if not wanted

% new command for cventry (this is done to allow users unbold or unitalicize the text in the cventry command)
\renewcommand*{\cventry}[6][.25em]{%
  \cvitem[#1]{#2}{%
    #3%
    \ifthenelse{\equal{#4}{}}{}{, #4}%
    \ifthenelse{\equal{#5}{}}{}{, #5}%
    \ifthenelse{\equal{#6}{}}{}{, #6}%
  }
}

\begin{document}
    \maketitle


    % save the original href command in a new command:
    \let\hrefWithoutArrow\href
     % new command for external links:
    \renewcommand{\href}[2]{\hrefWithoutArrow{#1}{\mbox{\color{color1} #2 \raisebox{.15ex}{\footnotesize \faExternalLink*}}}}

    \hypersetup{pdftitle={Kajetan Kołtoński's CV}}

    \section{Welcome to RenderCV!}

        \cvline{}{\href{https://github.com/sinaatalay/rendercv}{RenderCV} is a LaTeX-based CV/resume framework. It allows you to create a high-quality CV or resume as a PDF file from a YAML file, with \textbf{full Markdown syntax support} and \textbf{complete control over the LaTeX code}.}

        \cvline{}{The boilerplate content was inspired by \href{https://github.com/dnl-blkv/mcdowell-cv}{Gayle McDowell}.}


    
    \section{Quick Guide}

        \cvlistitem{Each section title is arbitrary and each section contains a list of entries.}

        \cvlistitem{There are 7 unique entry types: \textit{BulletEntry}, \textit{TextEntry}, \textit{EducationEntry}, \textit{ExperienceEntry}, \textit{NormalEntry}, \textit{PublicationEntry}, and \textit{OneLineEntry}.}

        \cvlistitem{Select a section title, pick an entry type, and start writing your section!}

        \cvlistitem{\href{https://docs.rendercv.com/user_guide/}{Here}, you can find a comprehensive user guide for RenderCV.}


    
    \section{Education}

        \cventry{Sept 2000 – May 2005}{\textbf{University of Pennsylvania}}{BS in Computer Science}{}{}{}
        \cvlistitem{GPA: 3.9/4.0 (\href{https://example.com}{a link to somewhere})}
        \cvlistitem{\textbf{Coursework:} Computer Architecture, Comparison of Learning Algorithms, Computational Theory}



    
    \section{Experience}

        \cventry{June 2005 – Aug 2007}{\textbf{Apple}}{Software Engineer}{Cupertino, CA}{}{}
        \cvlistitem{Reduced time to render user buddy lists by 75\% by implementing a prediction algorithm}
        \cvlistitem{Integrated iChat with Spotlight Search by creating a tool to extract metadata from saved chat transcripts and provide metadata to a system-wide search database}
        \cvlistitem{Redesigned chat file format and implemented backward compatibility for search}


        \cventry{June 2003 – Aug 2003}{\textbf{Microsoft}}{Software Engineer Intern}{Redmond, WA}{}{}
        \cvlistitem{Designed a UI for the VS open file switcher (Ctrl-Tab) and extended it to tool windows}
        \cvlistitem{Created a service to provide gradient across VS and VS add-ins, optimizing its performance via caching}
        \cvlistitem{Built an app to compute the similarity of all methods in a codebase, reducing the time from $\mathcal{O}(n^2)$ to $\mathcal{O}(n \log n)$}
        \cvlistitem{Created a test case generation tool that creates random XML docs from XML Schema}
        \cvlistitem{Automated the extraction and processing of large datasets from legacy systems using SQL and Perl scripts}



    
    \section{Publications}

        \cventry{Jan 2004}{\textbf{3D Finite Element Analysis of No-Insulation Coils}}{}{\href{https://doi.org/10.1109/TASC.2023.3340648}{10.1109/TASC.2023.3340648}}{}{}
        \cvline{}{\small \mbox{Frodo Baggins}, \mbox{\textbf{\textit{John Doe}}}, \mbox{Samwise Gamgee}}


    
    \section{Projects}

        \cventry{\href{https://github.com/sinaatalay/rendercv}{github.com/name/repo}}{Multi-User Drawing Tool}{}{}{}{}
        \cvlistitem{Developed an electronic classroom where multiple users can simultaneously view and draw on a "chalkboard" with each person's edits synchronized}
        \cvlistitem{Tools Used: C++, MFC}


        \cventry{\href{https://github.com/sinaatalay/rendercv}{github.com/name/repo}}{Synchronized Desktop Calendar}{}{}{}{}
        \cvlistitem{Developed a desktop calendar with globally shared and synchronized calendars, allowing users to schedule meetings with other users}
        \cvlistitem{Tools Used: C\#, .NET, SQL, XML}


        \cventry{2002}{Custom Operating System}{}{}{}{}
        \cvlistitem{Built a UNIX-style OS with a scheduler, file system, text editor, and calculator}
        \cvlistitem{Tools Used: C}



    
    \section{Technologies}

        \cvline{Languages}{C++, C, Java, Objective-C, C\#, SQL, JavaScript}

        \cvline{Technologies}{.NET, Microsoft SQL Server, XCode, Interface Builder}


    

\end{document}