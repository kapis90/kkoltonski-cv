%% start of file `template.tex'.
%% Copyright 2006-2015 Xavier Danaux (xdanaux@gmail.com), 2020-2022 moderncv maintainers (github.com/moderncv).
%
% This work may be distributed and/or modified under the
% conditions of the LaTeX Project Public License version 1.3c,
% available at http://www.latex-project.org/lppl/.

\documentclass[10pt,letterpaper,sans]{moderncv}        % possible options include font size ('10pt', '11pt' and '12pt'), paper size ('a4paper', 'letterpaper', 'a5paper', 'legalpaper', 'executivepaper' and 'landscape') and font family ('sans' and 'roman')

% moderncv themes
\moderncvstyle{classic}                            % style options are 'casual' (default), 'classic', 'banking', 'oldstyle' and 'fancy'
\moderncvcolor{blue}                               % color options 'black', 'blue' (default), 'burgundy', 'green', 'grey', 'orange', 'purple' and 'red'
%\renewcommand{\familydefault}{\sfdefault}         % to set the default font; use '\sfdefault' for the default sans serif font, '\rmdefault' for the default roman one, or any tex font name
\photo[3cm][1pt]{../profile.jpg}
\usepackage{amsmath} % for math

% adjust the page margins
\usepackage[scale=0.75]{geometry}
\setlength{\hintscolumnwidth}{3.8 cm}                % if you want to change the width of the column with the dates
%\setlength{\makecvheadnamewidth}{10cm}            % for the 'classic' style, if you want to force the width allocated to your name and avoid line breaks. be careful though, the length is normally calculated to avoid any overlap with your personal info; use this at your own typographical risks...

% font loading
% for luatex and xetex, do not use inputenc and fontenc
% see https://tex.stackexchange.com/a/496643
\ifxetexorluatex
  \usepackage{fontspec}
  \usepackage{unicode-math}
  \defaultfontfeatures{Ligatures=TeX}
  \setmainfont{Latin Modern Roman}
  \setsansfont{Latin Modern Sans}
  \setmonofont{Latin Modern Mono}
  \setmathfont{Latin Modern Math} 
\else
  \usepackage[T1]{fontenc}
  \usepackage{lmodern}
\fi

% document language
\usepackage[english]{babel}  % FIXME: using spanish breaks moderncv

% personal data
\name{Kajetan Kołtoński}{}
% \familyname{}
\address{Przeciszów}{}
\phone[mobile]{+48 537 648 594}
\email{k.koltonski@gmail.com}

\social[linkedin]{kajetan-koltonski}
% Social icons
% \social[linkedin]{john.doe}                        % optional, remove / comment the line if not wanted
% \social[xing]{john\_doe}                           % optional, remove / comment the line if not wanted
% \social[twitter]{ji\_doe}                          % optional, remove / comment the line if not wanted
% \social[github]{jdoe}                              % optional, remove / comment the line if not wanted
% \social[gitlab]{jdoe}                              % optional, remove / comment the line if not wanted
% \social[stackoverflow]{0000000/johndoe}            % optional, remove / comment the line if not wanted
% \social[bitbucket]{jdoe}                           % optional, remove / comment the line if not wanted
% \social[skype]{jdoe}                               % optional, remove / comment the line if not wanted
% \social[orcid]{0000-0000-000-000}                  % optional, remove / comment the line if not wanted
% \social[researchgate]{jdoe}                        % optional, remove / comment the line if not wanted
% \social[researcherid]{jdoe}                        % optional, remove / comment the line if not wanted
% \social[telegram]{jdoe}                            % optional, remove / comment the line if not wanted
% \social[whatsapp]{12345678901}                     % optional, remove / comment the line if not wanted
% \social[signal]{12345678901}                       % optional, remove / comment the line if not wanted
% \social[matrix]{@johndoe:matrix.org}               % optional, remove / comment the line if not wanted
% \social[googlescholar]{googlescholarid}            % optional, remove / comment the line if not wanted

% new command for cventry (this is done to allow users unbold or unitalicize the text in the cventry command)
\renewcommand*{\cventry}[6][.25em]{%
  \cvitem[#1]{#2}{%
    #3%
    \ifthenelse{\equal{#4}{}}{}{, #4}%
    \ifthenelse{\equal{#5}{}}{}{, #5}%
    \ifthenelse{\equal{#6}{}}{}{, #6}%
  }
}

\begin{document}
    \maketitle


    % save the original href command in a new command:
    \let\hrefWithoutArrow\href
     % new command for external links:
    \renewcommand{\href}[2]{\hrefWithoutArrow{#1}{\mbox{\color{color1} #2 \raisebox{.15ex}{\footnotesize \faExternalLink*}}}}

    \hypersetup{pdftitle={Kajetan Kołtoński's CV}}

    \section{Summary}

        \cvline{}{Senior Software Tester with experience in automation industry. Quality assurance leader for couple of years already.}

        \cvline{}{Mostly experienced with testing desktop application, but familiar with web applications too.}

        \cvline{}{Good attitude and motivation to work in group, creativity, and punctuality.}

        \cvline{}{Happy dad of two daughters and a son.}


    
    \section{Education}

        \cventry{2014}{\textbf{Silesian University of Technology}}{Bachelor in Electronics and Telecommunication}{}{}{}



    
    \section{Experience}

        \cventry{June 2022 – present}{\textbf{Rockwell Automation}}{Project Engineer Software Test / Test Architect}{}{}{}
        \cvlistitem{TODO}


        \cventry{Apr 2021 – May 2022}{\textbf{Rockwell Automation}}{Senior Engineer Software Test / Test Lead}{}{}{}
        \cvlistitem{TODO}


        \cventry{Mar 2019 – Mar 2021}{\textbf{Rockwell Automation}}{Senior Engineer Software Test}{}{}{}
        \cvlistitem{TODO}


        \cventry{Mar 2018 – Feb 2019}{\textbf{Rockwell Automation}}{Engineer Software Test}{}{}{}
        \cvlistitem{TODO}


        \cventry{Aug 2016 – Feb 2018}{\textbf{Mentor Graphics}}{Quality Assurance Engineer}{}{}{}
        \cvlistitem{TODO}


        \cventry{Aug 2015 – July 2016}{\textbf{Mentor Graphics}}{Associate Quality Assurance Engineer}{}{}{}
        \cvlistitem{TODO}


        \cventry{Oct 2014 – July 2015}{\textbf{Mentor Graphics}}{Product Quality Assurance Intern}{}{}{}
        \cvlistitem{TODO}



    
    \section{Technologies}

        \cvline{Languages}{Python (Advanced), TCL}

        \cvline{Technologies}{Pytest, Squish/Selenium framework, Jira, qTest, Jenkins, GitLab, GIT / SVN, SCRUM/Agile work methodology, Windows / Linux}


    
    \section{Languages}

        \cvline{}{English - very good knowledge in speaking and writing}

        \cvline{}{German - very basic knowledge}


    
    \section{RODO}

        \cvline{}{Wyrażam zgodę na przetwarzanie moich danych osobowych dla potrzeb niezbędnych do realizacji procesu rekrutacji zgodnie z Rozporządzeniem Parlamentu Europejskiego i Rady (UE) 2016/679 z dnia 27 kwietnia 2016 r. w sprawie ochrony osób fizycznych w związku z przetwarzaniem danych osobowych i w sprawie swobodnego przepływu takich danych oraz uchylenia dyrektywy 95/46/WE (RODO).}


    

\end{document}