%% start of file `template.tex'.
%% Copyright 2006-2015 Xavier Danaux (xdanaux@gmail.com), 2020-2022 moderncv maintainers (github.com/moderncv).
%
% This work may be distributed and/or modified under the
% conditions of the LaTeX Project Public License version 1.3c,
% available at http://www.latex-project.org/lppl/.

\documentclass[10pt,letterpaper,sans]{moderncv}        % possible options include font size ('10pt', '11pt' and '12pt'), paper size ('a4paper', 'letterpaper', 'a5paper', 'legalpaper', 'executivepaper' and 'landscape') and font family ('sans' and 'roman')

% moderncv themes
\moderncvstyle{classic}                            % style options are 'casual' (default), 'classic', 'banking', 'oldstyle' and 'fancy'
\moderncvcolor{blue}                               % color options 'black', 'blue' (default), 'burgundy', 'green', 'grey', 'orange', 'purple' and 'red'
%\renewcommand{\familydefault}{\sfdefault}         % to set the default font; use '\sfdefault' for the default sans serif font, '\rmdefault' for the default roman one, or any tex font name
\photo[3cm][1pt]{../profile.jpg}
\usepackage{amsmath} % for math

% adjust the page margins
\usepackage[scale=0.75]{geometry}
\setlength{\hintscolumnwidth}{3.8 cm}                % if you want to change the width of the column with the dates
%\setlength{\makecvheadnamewidth}{10cm}            % for the 'classic' style, if you want to force the width allocated to your name and avoid line breaks. be careful though, the length is normally calculated to avoid any overlap with your personal info; use this at your own typographical risks...

% font loading
% for luatex and xetex, do not use inputenc and fontenc
% see https://tex.stackexchange.com/a/496643
\ifxetexorluatex
  \usepackage{fontspec}
  \usepackage{unicode-math}
  \defaultfontfeatures{Ligatures=TeX}
  \setmainfont{Latin Modern Roman}
  \setsansfont{Latin Modern Sans}
  \setmonofont{Latin Modern Mono}
  \setmathfont{Latin Modern Math} 
\else
  \usepackage[T1]{fontenc}
  \usepackage{lmodern}
\fi

% document language
\usepackage[english]{babel}  % FIXME: using spanish breaks moderncv

% personal data
\name{Kajetan Kołtoński}{}
% \familyname{}
\address{Przeciszów}{}
\phone[mobile]{+48 537 648 594}
\email{k.koltonski@gmail.com}

\social[linkedin]{kajetan-koltonski}
% Social icons
% \social[linkedin]{john.doe}                        % optional, remove / comment the line if not wanted
% \social[xing]{john\_doe}                           % optional, remove / comment the line if not wanted
% \social[twitter]{ji\_doe}                          % optional, remove / comment the line if not wanted
% \social[github]{jdoe}                              % optional, remove / comment the line if not wanted
% \social[gitlab]{jdoe}                              % optional, remove / comment the line if not wanted
% \social[stackoverflow]{0000000/johndoe}            % optional, remove / comment the line if not wanted
% \social[bitbucket]{jdoe}                           % optional, remove / comment the line if not wanted
% \social[skype]{jdoe}                               % optional, remove / comment the line if not wanted
% \social[orcid]{0000-0000-000-000}                  % optional, remove / comment the line if not wanted
% \social[researchgate]{jdoe}                        % optional, remove / comment the line if not wanted
% \social[researcherid]{jdoe}                        % optional, remove / comment the line if not wanted
% \social[telegram]{jdoe}                            % optional, remove / comment the line if not wanted
% \social[whatsapp]{12345678901}                     % optional, remove / comment the line if not wanted
% \social[signal]{12345678901}                       % optional, remove / comment the line if not wanted
% \social[matrix]{@johndoe:matrix.org}               % optional, remove / comment the line if not wanted
% \social[googlescholar]{googlescholarid}            % optional, remove / comment the line if not wanted

% new command for cventry (this is done to allow users unbold or unitalicize the text in the cventry command)
\renewcommand*{\cventry}[6][.25em]{%
  \cvitem[#1]{#2}{%
    #3%
    \ifthenelse{\equal{#4}{}}{}{, #4}%
    \ifthenelse{\equal{#5}{}}{}{, #5}%
    \ifthenelse{\equal{#6}{}}{}{, #6}%
  }
}

\begin{document}
    \maketitle


    % save the original href command in a new command:
    \let\hrefWithoutArrow\href
     % new command for external links:
    \renewcommand{\href}[2]{\hrefWithoutArrow{#1}{\mbox{\color{color1} #2 \raisebox{.15ex}{\footnotesize \faExternalLink*}}}}

    \hypersetup{pdftitle={Kajetan Kołtoński's CV}}

    \section{Summary}

        \cvline{}{Senior Software Tester with experience in the automation industry. Quality assurance leader for a couple of years already.}

        \cvline{}{Mostly experienced with testing desktop applications, but familiar with web applications too.}

        \cvline{}{Good attitude and motivation to work in group, creativity, and punctuality.}

        \cvline{}{Happy dad of two daughters and a son.}


    
    \section{Education}

        \cventry{2014}{\textbf{Silesian University of Technology}}{Bachelor in Electronics and Telecommunication}{}{}{}



    
    \section{Experience}

        \cventry{June 2022 – present}{\textbf{Rockwell Automation}}{\emph{FactoryTalk FTOptix}}{\newline{}Project Engineer Software Test / Test Architect}{}{}
        \cvlistitem{Build core architecture of test automation framework}
        \cvlistitem{Coordinate test automation work for the entire project on different campuses located in the United States, Poland, Italy, and China.}
        \cvlistitem{Integrate GitLab CI/CD with automation framework}
        \cvlistitem{Act as \emph{Product Owner} for test automation team}


        \cventry{Mar 2018 – May 2022}{\textbf{Rockwell Automation}}{\emph{Studio 5000 View Designer}}{\newline{}Senior Engineer Software Test / Test Lead}{}{}
        \cvlistitem{Create test cases, both manual and automated}
        \cvlistitem{Run and analyze regression tests}
        \cvlistitem{Perform performance tests}
        \cvlistitem{Contribute to test automation framework}
        \cvlistitem{Act as \emph{Feaure Test Lead} to confirm that feature is properly tested}
        \cvlistitem{Become a technical leader of the test team, \emph{Apr 2021}}
        \cvlistitem{Work using SAFe methodology in one of 11 Feature Teams\newline{}\newline{}\newline{}}


        \cventry{Oct 2014 – Feb 2018}{\textbf{Mentor Graphics}}{\emph{Expedition Package Integrator}}{\newline{}Quality Assurance Engineer}{}{}
        \cvlistitem{Begin my career as an intern and decide to continue the quality assurance path}
        \cvlistitem{Create test cases and test strategies for currently developed features}
        \cvlistitem{Create automated test cases}
        \cvlistitem{Run and analyze regression tests}



    
    \section{Technologies}

        \cvline{Languages}{Python (Advanced), TCL}

        \cvline{Technologies}{Pytest, Squish, Selenium, Jira, qTest, Jenkins, GitLab CI/CD, GIT / SVN, SCRUM/SAFe, Windows / Linux}


    
    \section{Languages}

        \cvline{}{English - excellent in speaking and writing}

        \cvline{}{German - very basic level}


    
    \section{RODO}

        \cvline{}{Wyrażam zgodę na przetwarzanie moich danych osobowych dla potrzeb niezbędnych do realizacji procesu rekrutacji zgodnie z Rozporządzeniem Parlamentu Europejskiego i Rady (UE) 2016/679 z dnia 27 kwietnia 2016 r. w sprawie ochrony osób fizycznych w związku z przetwarzaniem danych osobowych i w sprawie swobodnego przepływu takich danych oraz uchylenia dyrektywy 95/46/WE (RODO).}


    

\end{document}